\documentclass{article}

\usepackage{fontspec}
\usepackage{ragged2e}
\usepackage{url}
\usepackage{polski}
\usepackage[margin=1in]{geometry}
\usepackage{amsmath}
\usepackage{amsthm}
\usepackage{graphicx}
\usepackage[colorinlistoftodos]{todonotes}
\usepackage[hidelinks]{hyperref}
\usepackage{amssymb}

\usepackage{listings}

\lstset{language=C}

\lstdefinestyle{customc}{
  belowcaptionskip=1\baselineskip,
  breaklines=true,
  frame=L,
  xleftmargin=\parindent,
  language=C,
  showstringspaces=false,
  basicstyle=\footnotesize\ttfamily,
  keywordstyle=\bfseries\color{green!40!black},
  commentstyle=\itshape\color{purple!40!black},
  identifierstyle=\color{blue},
  stringstyle=\color{orange},
}

\lstset{escapechar=@,style=customc}

\title{Pwn-ing Linux x86/x64 \\ Zadania kwalifikacyjne}
\author{Jakub Szewczyk \\
Grzegorz Uriasz}

\begin{document}
\maketitle
\normalsize

\section{Korzystanie z Windowsa}
Nie polecamy korzystania z Windowsa, ponieważ na naszych warsztatach skupiamy się na exploitacji programów działających
na Linuxie. Dlatego zachęcamy do zainstalowania dystrybucji Linuxa na swoim komputerze w konfiguracji dual boot(wybór 
systemu przy starcie komputera), np. Linux Mint, Antergos, Ubuntu. Inną opcją jest zainstalowanie modułu WSL na Windowsie,
znanego też jako Bash on Windows - symuluje on działające jądro Linuxa na Windowsie 10, jest oficjalnym modułem Microsoftu
i powinien wystarczyć do naszych zadań. Instrukcje jak go zainstalować są tu: \url{https://msdn.microsoft.com/en-us/commandline/wsl/install_guide}. Polecamy wcześniej zaktualizować system do wydania Creator's update, WSL został
tam znacznie ulepszony w stosunku do Anniversary Update.\\
Potrzebne będzie zainstalowanie pakietów w systemach Ubuntu, Mint, bądź WSL komendą:
\texttt{sudo apt install libssl-dev gcc-multilib yasm}

\section{Dlaczego chcesz uczestniczyć w tych warsztatach? - Obowiązkowo}
Odpowiedz na to pytanie ściśle i zwięźle. Nie jest to zadanie typu "Gdybyś był owocem, to jakim? Uzasadnij twoje stanowisko na minimalnie dwudziestu stronach A4" - oczekujemy tylko krótkiego opisu.

\section{Rozgrzewka - reprezentacja binarna liczb zmiennoprzecinkowych}
Napisz program, który wykonuje dwie operacje (język programowania dowolny, prosimy o kod źródłowy):
\begin{enumerate}
\item Wczytuje liczbę zmiennoprzecinkową, a następnie wypisuje jej reprezentację bitową w pamięci komputera jako liczbę zmiennoprzecinkową podwójnej precyzji(double) oraz pojedynczej precyzji(float).
\item Wczytuje reprezentacje binarną liczby zmiennoprzecinkowej podwójnej precyzji w formacie litle-endian(tak jak w poprzednim podpunkcie na PC-tach się wypisze) oraz dla niej wypisuje odpowiadającą jej wartość liczbową w systemie dziesiętnym.
\end{enumerate}
% Na obrazku był błąd
% Przykładowe wywołanie programu może przebiegać następująco: \\ \\
% \includegraphics[height=3.3cm]{przebieg.png}

\section{Assembler -- Instalacja i Przetestowanie}
Polecamy assembler yasm (\url{http://yasm.tortall.net/Download.html}), ale zadziała także nasm, fasm, lub jakikolwiek
inny, jeżeli znasz jego składnię. Zainstaluj assembler, zassembluj podany poniżej program i go uruchom na Linuxie bądź WSLu,
pokaż screenshoty. Nie przejmuj się, jeżeli nie rozumiesz tego kodu -- chodzi tylko o sprawdzenie narzędzia,
podstawy assemblera wyjaśnimy na początku warsztatów.
\subsection{Wersja 32-bitowa - nie działa pod WSL}
\begin{lstlisting}[frame=single]
;; hello.asm
segment .data
    msg     db      "Hello WWW13!", 0Ah

segment .text
        global  _start
_start:
 ; write
    mov     eax, 4
    mov     ebx, 1
    mov     ecx, msg
    mov     edx, 13
    int     80h
 ; exit
    mov     eax, 1
    xor     ebx, ebx
    int     0x80
\end{lstlisting}
Komenda do zassemblowania: \texttt{yasm -f elf32 -o hello.o hello.asm} \\
Komenda do zlinkowania: \texttt{ld hello.o -o hello} \\
Komenda do uruchomienia: \texttt{./hello} \\
\subsection{Wersja 64-bitowa - działa pod WSL}
\begin{lstlisting}[frame=single]
;; hello64.asm
segment .data
    msg     db      "Hello WWW13!", 0Ah

segment .text
        global  _start
_start:
 ; write
    mov     rax, 1
    mov     rdi, 1
    mov     rsi, msg
    mov     rdx, 13
    syscall
 ; exit
    mov     rax, 60
    xor     rdi, 0
    syscall
\end{lstlisting}
Komenda do zassemblowania: \texttt{yasm -f elf64 -o hello64.o hello64.asm} \\
Komenda do zlinkowania: \texttt{ld hello64.o -o hello64} \\
Komenda do uruchomienia: \texttt{./hello64} \\

\section{SSH -- Instalacja i Przetestowanie}
Zainstaluj SSH (zobacz opis warsztatów) i prześlij nam zrzut ekranu, co się wyświetli jak połączysz się z github.com na
koncie git (komenda: \texttt{ssh} \path{git@github.com}, należy zaakceptować certyfikat githuba).
Jeżeli wyświetli się permission denied, należy sprawdzić czy wygenerowało się swój klucz oraz czy dodało się go do
konta na githubie (które warto założyć, bo nawet jeżeli na tych warsztatach się nie przyda, to później na pewno).

\section{Operacje logiczne}
Odpowiedz na pytania i rozwiąż poniższe równania.
\subsection{W jaki sposób można odejmować używając dodawania?}
Dodawanie na bramkach logicznych jest o wiele łatwiejsze do zrealizowania niż odejmowanie. A więc w jaki sposób używając operacji logicznych typu and, xor, or ... uzyskać odejmowanie dwóch liczb binarnych mając układ dodający dwie liczby binarne do siebie? \\
Porada: Poczytaj o reprezentacjach binarnych liczb ze znakiem.
\subsection{Operacje bitowe}
Masz do dyspozycji 8-bitową zmienną liczbową bez znaku o nazwie Adam. Zaproponuj działania dzięki których ustawisz i-ty bit Adama na 1 oraz j-ty bit Adama na 0. W jaki sposób sprawdzić czy n-ty bit Adama jest jedynką?

Oblicz na kartce(bez pomocy komputera) poniższe dwa działania:\\
((24 and 101) or (132 xor 241)) nor 5 = ? \\
(123 xor 153) xor (246 and 235) = ? \\

\section{Błąd w kodzie C}
Poniższy kod zawiera wywołanie funkcji system która otwiera powłokę systemową. Na pierwszy rzut oka dotarcie do tej linijki kodu jest niemożliwe ze względu na sprawdzenie, czy pole casual w strukturze state jest różne od 1. Otóż poniższy kod w języku C zawiera błąd który pozwala przejąć Tobie kontrolę nad polem casual. Twoim zadaniem jest przeanalizowanie kodu, znalezienie wspomnianego błędu i opisanie nam go w rozwiązaniu. W rozwiązaniu podaj przykładowe wejście które pozwala na wywołanie funkcji system oraz sposób naprawy tego błędu.

\begin{lstlisting}[frame=single, language=c]
#include <stdio.h>
#include <stdlib.h>

typedef struct
{
	char tab[100];
	char casual;
} state;

state s;

int main()
{
	int ans, i;
	printf("How many times to loop: ");
	scanf("%d", &ans);
	if(ans<0 || ans>100)
	{
		printf("Invalid number\n");
		return 0;
	}

	s.casual = 1;
	
	for(i = 0; i<=ans; i++)
	{
		s.tab[i] = i;
	}

	if(s.casual != 1)
	{
		printf("YOU SHALL NOT CALL ME!\n");
		system("/bin/sh");
		return 0;
	}
	printf("No shell for you\n");

	return 0;
}

\end{lstlisting}


\section{Obfuskacja kodu w C - Niewymagane, ale polecamy spróbować}
Zapomniałem hasła do Facebooka! Na szczęście mam zapisane to hasło na komputerze, więc teraz fakt że mam jeden port ethernet na pewno mi nie przeszkodzi (nie wiem czemu ssh na serwerze z moją kopią zapasową wymaga komputera z aż 22 portami ethernet!). Problem w tym, że analizowałem nowego wirusa krążącego po Internecie i przez przypadek go uruchomiłem. Zaszyfrował mi wszystkie pliki! Dobra wiadomość jest taka, że udało mi się uzyskać fragment kodu źródłowego wirusa, który jest odpowiedzialny za sprawdzenie, czy hasło deszyfrujące jest dobre. Ja nie mogę przeanalizować tego kodu, bo mam zaszyfrowanego vim-a, więc potrzebuję twojej pomocy. Na szczęście kod został napisany w języku C, którego składnia jest najpiękniejsza na świecie i kody są najczytelniejsze na świecie, więc nie powinieneś mieć z jego analizą najmnniejszego problemu.

\begin{lstlisting}[frame=single, language=c]
//0bfu5c473d p455w0rd ch3ck3r
#include <stdio.h>
#include <stdlib.h>
#include <string.h>
#include <openssl/md5.h>
char *_____ = "\x66\x65\x64\x05\x06\x4d\x58\x53\x56\x5f\x4d\x00\x31\x17\x0b\
\x6b\x61\x6d\x08\x32\x25\x2c\x3a\x09\x17\x00\x24\x0c\x4d";
const unsigned char* ___ = "\x43\x2f\x45\xb4\x4c\x43\x24\x14\xd2\xf9\x7d\xf0\
\xe5\x74\x38\x18\x36\x04\x17\x76\x71\x22\x2a\x37\x76\x36\x3e\x46\x06\x0b\x77\
\x74\x28\x3b\x27\x77\x55\x4d";
char ____[100]; unsigned char __[MD5_DIGEST_LENGTH+1]; char _;
void main(){memset(____, 0, sizeof(____));printf("Enter The Password: ");
  scanf("%20s", ____);int _______ = strlen(____);int ______ = strlen(_____);
  for(((1&2)??(&_??))??'=((5&1)??(&_-('c'^'b')??));_<_______;)((_??(____??))??'=
  ((_%______)??(_____??)), (3??(&_-3??))++);MD5((const unsigned char*)____, _______,
   (__??(MD5_DIGEST_LENGTH??)=0,__));int ________ = strlen(_____+______+1)+3;
  for((6??(&_-6??))^=(7??(&_-7??)); _<________;)(____??(_??) = (_____??(_+______+1??)^
  _____??(_%______??))),(4[&_-4])++;________ = strlen(___+MD5_DIGEST_LENGTH);
  for((8??(&_-8??))^=(9??(&_-9??)); _<________;)(____??(_+strlen(____)+1??)=
  ((___+MD5_DIGEST_LENGTH)??(_??))^(_____??(_%______??))),(5??(&_-5??))++;
  for(((1&3)??(&_-1??))??'=(((2&4))??(&_??)); _<MD5_DIGEST_LENGTH;)(_??(___??)??'
  (_??(__??))?(printf(____),exit(0)):(2??(&_-2??))++);printf(____+strlen(____)+1);
  exit(0);}

\end{lstlisting}
Kompilacja programu odbywa się poleceniem: 
\begin{lstlisting}[frame=single, language=bash]
gcc -o <nazwa pliku wyjsciowego> <kod zrodlowy> -l ssl -l crypto --trigraphs
\end{lstlisting}

Rozwiązaniem zadania jest:
\begin{enumerate}
\item Co zrobiłeś by zrozumieć powyższy kod?
\item W jaki sposób powyższy kod sprawdza hasło?
\item Podaj listę metod zaciemniania zastosowanych w powyższym kodzie.
\item Podaj hasło które po wczytaniu przez program jest indentyfikowane jako poprawne.
\end{enumerate}

\normalsize

\centering
\LARGE{Powodzenia i do zobaczenia w Wierchomli!}

\end{document}